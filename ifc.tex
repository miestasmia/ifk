\documentclass[a4paper,class=article,tikz,border=0pt]{standalone}

\usepackage[pass,a4paper]{geometry}
\usepackage[utf8]{inputenc}
\usepackage{tikz}
\usepackage{anyfontsize}
\usepackage{arrayjob}
\usepackage{calculator}
\usepackage{ifthen}
\usepackage{datenumber}
\usepackage{minibox}

\input{datenumberesperanto.ldf}

\newarray\semajno
\readarray{semajno}{dimanĉo&lundo&mardo&merkredo&ĵaŭdo&vendredo&sabato}

\newcommand\xsize[2]{\fontsize{#1}{
    \MULTIPLY{#1}{1.2}{\distanco}
    \distanco
}\selectfont#2}
\newcommand\kromtago[1]{
    \node at (277/7*6 mm + 10mm + 277mm / 2 / 7, 204mm - 135/4mm) {\Large\textbf{#1}};
    \node [anchor=west] at (13mm + 277/7*6 mm, 194mm - 135/4mm) {\Large 29};
    \draw[gray] (277/7*6 mm + 10mm, 200mm - 135/4 mm) -- (287mm, 200mm - 135/4 mm);

    % Ĉu estas ĉiujara kalendaro?
    \ifthenelse{\equal{\jaro}{}}{}{
        \node[gray,anchor=west] at (277/7*6 mm + 13mm, 168mm - 135/4mm) {\thedateday-a de  \datemonthname};
        \nextdate
    }
}

% Ŝanĝu ĉi tion al nenio por havi ĉiujaran kalendaron. Alie, ŝanĝu al la jaro de la kalendaro
\newcommand\jaro{}

\newcounter{semajnnumero}

\begin{document}
    \fontfamily{qag}\selectfont

    % Se ne malplena
    \ifthenelse{\equal{\jaro}{}}{}{
        \setdate{\jaro}{1}{1}
    }

    \foreach \monato in {Januaro,Februaro,Marto,Aprilo,Majo,Junio,Sunio,Julio,Aŭgusto,Septembro,Oktobro,Novembro,Decembro}{
        \begin{tikzpicture}
            % Fono
            \draw[transparent] (0,0) rectangle (297mm,210mm);

            % Nomo de la monato, eble jaro kaj nomo de la kalendaro
            \node [anchor=north west,align=left] at (10mm,190mm) {
                \xsize{80}{\monato\ \jaro}
                \minibox{
                    \\[-.25em]
                    \xsize{20}{Internacia} \\[.4em]
                    \xsize{20}{Fiksita} \\[.4em]
                    \xsize{20}{Kalendaro}
                }
            };

            % LINIOJ
            % Vertikalaj linioj
            \foreach \x in {0,...,7} {
                %\draw[gray] (10mm + 277/7*\x mm, 145mm) -- (10mm + 277/7*\x mm, 10mm);
            }

            \foreach \y in {0,...,4} {
                % Horizontalaj linioj
                \draw[gray] (10mm, 10mm + 135mm - 135/4*\y mm) -- (287mm, 10mm + 135mm - 135/4*\y mm);

                % Numero de la semajno
                \ifthenelse{\y>0}{
                    \stepcounter{semajnnumero}
                    \draw[gray,fill=white] (11mm, 10mm + 1.25*135mm - 135/4*\y mm) circle[radius=3mm] node {\thesemajnnumero};
                }{}
            }

            % ALIAJ INFORMOJ
            % Tago de la semajno
            \foreach \x in {1,...,7} {
                \node at (10mm + 277/7*\x mm - 277mm / 2 / 7, 150mm) {\Large\textbf{\semajno(\x)}};
            }

            \foreach \n in {0,...,27} {
                % Tago de la monato
                \ADD{\n}{1}{\N}
                \INTEGERDIVISION{\n}{7}{\line}{\xmod}
                \node [anchor=west] at (15mm + 277/7*\xmod mm, 140mm - 135/4*\line mm) {\large\N};

                % Ekvivalenta Gregoria tago
                \ifthenelse{\equal{\jaro}{}}{}{
                    \node[gray,anchor=west] at (15mm + 277/7*\xmod mm, 115mm - 135/4*\line mm) {\thedateday-a de \datemonthname};
                    \nextdate
                }
            }

            % KROMTAGOJ
            % Supertago
            \ifthenelse{\equal{\monato}{Junio}}{
                % Ĉu estas superjaro
                \MODULO{\thedateyear}{4}{\xmodUNU}
                \MODULO{\thedateyear}{100}{\xmodDU}
                \MODULO{\thedateyear}{400}{\xmodTRI}
                \ifthenelse{\equal{\xmodUNU}{0}\AND\NOT\equal{\xmodDU}{0}\OR\equal{\xmodTRI}{0}}{
                    % Jes
                    \kromtago{supertago}
                }{
                    % Ne
                    % Sed ĉu estas ĉiujara kalendaro?
                    \ifthenelse{\equal{\jaro}{}}{
                        \kromtago{supertago}
                    }{}
                }
            }{}

            % Jartago
            \ifthenelse{\equal{\monato}{Decembro}}{
                \kromtago{jartago}
            }{}
        \end{tikzpicture}
    }
\end{document}
